%%%%%%%%%%%%%%%%%%%%%%%%%%%%%%%%%%%%%%%%%
% "ModernCV" CV and Cover Letter
% LaTeX Template
% Version 1.1 (9/12/12)
%
% This template has been downloaded from:
% http://www.LaTeXTemplates.com
%
% Original author:
% Xavier Danaux (xdanaux@gmail.com)
%
% License:
% CC BY-NC-SA 3.0 (http://creativecommons.org/licenses/by-nc-sa/3.0/)
%
% Important note:
% This template requires the moderncv.cls and .sty files to be in the same 
% directory as this .tex file. These files provide the resume style and themes 
% used for structuring the document.
%
%%%%%%%%%%%%%%%%%%%%%%%%%%%%%%%%%%%%%%%%%

%----------------------------------------------------------------------------------------
%   PACKAGES AND OTHER DOCUMENT CONFIGURATIONS
%----------------------------------------------------------------------------------------

\documentclass[11pt,a4paper,sans]{moderncv} % Font sizes: 10, 11, or 12; paper sizes: a4paper, letterpaper, a5paper, legalpaper, executivepaper or landscape; font families: sans or roman

\moderncvstyle{casual} % CV theme - options include: 'casual' (default), 'classic', 'oldstyle' and 'banking'
\moderncvcolor{blue} % CV color - options include: 'blue' (default), 'orange', 'green', 'red', 'purple', 'grey' and 'black'

\usepackage{fontawesome}
\usepackage[scale=0.8]{geometry} % Reduce document margins
%\setlength{\hintscolumnwidth}{3cm} % Uncomment to change the width of the dates column
%\setlength{\makecvtitlenamewidth}{10cm} % For the 'classic' style, uncomment to adjust the width of the space allocated to your name

%----------------------------------------------------------------------------------------
%   NAME AND CONTACT INFORMATION SECTION
%----------------------------------------------------------------------------------------

\firstname{Trever} % Your first name
\familyname{Hines} % Your last name
\address{2534 CC Little Bldg}{1100 North University Ave}{Ann Arbor, MI 48109-1005}
\phone{(217) 766 4445}
\social[github]{treverhines}
\email{hinest@umich.edu}

%----------------------------------------------------------------------------------------

\begin{document}

\makecvtitle % Print the CV title

\section{Education}
\cvitem{2012--present}{\textbf{PhD candidate Geophysics (certificate in Computational Engineering)}
                       \newline \textit{University of Michigan}, Ann Arbor}
                       
\cvitem{2008-2012}{\textbf{BS Geology}, \textit{University of Illinois}, Urbana-Champaign}

\section{Doctoral Dissertation}
\cvitem{Title}{Transient ground deformation in tectonically active regions and implications for the mechanical behavior of the crust and upper mantle}

\cvitem{Advisor}{Eric Hetland}

\cvitem{Description}{In this dissertation I introduce novel techniques to identify and model transient ground deformation resulting from earthquakes.  My work further constrains the physical properties of the crust and upper mantle, which are key pieces of information needed to improve assessments of seismic hazard.}

\section{Publications}
\cvitem{2017}{Hines, T. T., and E. A. Hetland. Revealing transient strain in geodetic data with Gaussian process regression. \textit{in preparation}}

\cvitem{2017}{Hines, T. T., and E. A. Hetland. Unbiased characterization of noise in geodetic data. \textit{submitted to J. Geod.}}

\cvitem{2016}{Hines, T. T. and E. A. Hetland. Rheologic constraints on the upper mantle from five years of postseismic deformation following the El Mayor-Cucapah earthquake. \textit{J. Geophys. Res.}, 121, doi: 10.1002/2016JB013114}
\cvitem{2016}{Hines, T. T. and E. A. Hetland. Rapid and simultaneous estimation of fault slip and heterogeneous lithospheric viscosity from post-seismic deformation. \textit{Geophys. J. Int.}, 204, doi: 10.1093/gji/ggv477}

\cvitem{2013}{Hines, T. T. and E. A. Hetland. Bias in estimates of lithosphere viscosity from interseismic deformation. \textit{Geophys. Res. Lett.}, 40, doi:10.1002/grl.50839}

\section{Research Positions}
\cvitem{2012--present}{\textbf{Graduate student research assistant}, \textit{University of Michigan}}

\cvitem{2011}{\textbf{Ecological modeling intern}, \textit{Smithsonian Environmental Research Center}}

\section{Teaching Experience}
\cvitem{2014--2016}{\textbf{Graduate student instructor}, \textit{University of Michigan}
                       \newline Taught labs for EARTH 468, Data Analysis and Model Estimation}
        
\cvitem{2015}{\textbf{Michigan Math and Science Scholars instructor}, \textit{University of Michigan}
               \newline Co-taught a summer course for high school students on the mathematics of natural hazard}
\newpage
\section{Honors}
\cvitem{2015}{\textbf{Best Graduate Student Instructor Award}, \textit{University of Michigan}}

\cvitem{2012}{\textbf{Outstanding Senior Award}, \textit{University of Illinois}}

\section{Workshops and Field Courses}
\cvitem{2014}{\textbf{CIG Crustal Deformation Modeling Workshop}, \textit{Stanford University}
              \newline One week course on using Pylith, a finite element program for modeling lithosphere dynamics}
              
\cvitem{2014}{\textbf{InSAR: An Introduction to ISCE and GIAnT}, \textit{UNAVCO}
              \newline Three day course on processing Interferometric Synthetic Aperture Radar (InSAR) data}
              
\cvitem{2012}{\textbf{Wasatch-Uinta Summer Field Camp}, \textit{University of Illinois}
              \newline Six week field mapping course in Utah}

\cvitem{2012}{\textbf{The Geology of County Clare, Western Eire}, \textit{University of Illinois}
              \newline Two week course studying the evolution of a sedimentary basin in Ireland}

\section{Abstracts}
\cvitem{2016}{Hines, T. T. and E. A. Hetland. Extracting transient strain from GPS: a novel application of the radial basis function-finite dfference method. American Geophysical Union 2016 Fall Meeting, San Francisco, CA.}

\cvitem{2016}{E. A. Hetland, L. M. Luna, R. H. Styron, and T. T. Hines. Inference of stress from faulting and topographic loading on faults. Seismological Society of America 2016 Annual Meeting, Reno, NV.}

\cvitem{2015}{Hines, T. T. and E. A. Hetland (invited speaker). Kalman filter based estimation of lithospheric viscosity and fault slip from postseismic deformation: application to the 2010 El Mayor-Cucapah earthquake. American Geophysical Union 2015 Fall Meeting, San Francisco, CA.}

\cvitem{2015}{Hines, T. T. and E. A. Hetland. Inversion of postseismic deformation for lithospheric viscosity and fault slip. Society for Industrial and Applied Mathematics: Geosciences 2015 Meeting, Stanford CA.}

\cvitem{2014}{Hines, T. T. and E. A. Hetland. Direct inversion of postseismic deformation for lithospheric viscosity structure and fault slip. American Geophysical Union 2014 Fall Meeting, San Francisco, CA.}

\cvitem{2014}{Hines, T. T. and E. A. Hetland. Determination of lithosphere rheology from interseismic deformation and implications for fault stress accumulation. Seismological Society of America 2014 Annual Meeting, Anchorage, AK.}

\cvitem{2013}{Hines, T. T. and E. A. Hetland. Bias in estimates of lithosphere viscosity from interseismic deformation. American Geophysical Union 2013 Fall Meeting, San Francisco, CA.}

\cvitem{2013}{Hetland, E. A., G. Zhang, and T. T. Hines. Post- and interseismic deformation due to both localized and distributed creep at depth. American Geophysical Union 2013 Fall Meeting, San Fancisco, CA.}

\cvitem{2013}{Hines, T. T. and E. A. Hetland. Evaluating geodetic constraints on the strength of the lithosphere. Michigan Geophysical Union, Ann Arbor, MI.}

\newpage
\section{Membership}
\cvitem{since 2013}{Seismological Society of America}
\cvitem{since 2013}{American Geophysical Union}
\cvitem{since 2012}{Society for Industrial and Applied Mathematics}


\end{document}