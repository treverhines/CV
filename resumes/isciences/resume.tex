%%%%%%%%%%%%%%%%%%%%%%%%%%%%%%%%%%%%%%%%%
% "ModernCV" CV and Cover Letter
% LaTeX Template
% Version 1.1 (9/12/12)
%
% This template has been downloaded from:
% http://www.LaTeXTemplates.com
%
% Original author:
% Xavier Danaux (xdanaux@gmail.com)
%
% License:
% CC BY-NC-SA 3.0 (http://creativecommons.org/licenses/by-nc-sa/3.0/)
%
% Important note:
% This template requires the moderncv.cls and .sty files to be in the same 
% directory as this .tex file. These files provide the resume style and themes 
% used for structuring the document.
%
%%%%%%%%%%%%%%%%%%%%%%%%%%%%%%%%%%%%%%%%%

%----------------------------------------------------------------------------------------
%   PACKAGES AND OTHER DOCUMENT CONFIGURATIONS
%----------------------------------------------------------------------------------------

\documentclass[11pt,a4paper,sans]{moderncv} 

\moderncvstyle{casual} 
\moderncvcolor{blue} 

\usepackage[scale=0.78]{geometry}
%\setlength{\hintscolumnwidth}{3cm} % Uncomment to change the width of the dates column

%----------------------------------------------------------------------------------------
%   NAME AND CONTACT INFORMATION SECTION
%----------------------------------------------------------------------------------------

\firstname{Trever T.} % Your first name
\familyname{Hines} % Your last name
\address{2534 CC Little Bldg}{1100 North University Ave}{Ann Arbor, MI 48109-1005}
\phone{(217) 766 4445}
\social[github]{treverhines}
\email{hinest@umich.edu}

%----------------------------------------------------------------------------------------
\begin{document}

\makecvtitle % Print the CV title
\vspace*{-1cm}
% Stuff to say
% Kalman filtering, to handle real-time data integration
% Radial Basis functions and stochastic modeling, for flexible nonparametric modeling 

\section{Education}
\cvitem{2012--present}{\textbf{PhD candidate Geophysics (certificate in Computational Engineering)}
                       \newline \textit{University of Michigan}, Ann Arbor, MI}
                       
\cvitem{2008-2012}{\textbf{BS Geology}, \textit{University of Illinois}, Urbana-Champaign, IL}

\section{Research Experience}

\cvitem{2012--present}{\textbf{Graduate student research assistant}, \textit{University of Michigan}
			          \begin{itemize}
                       \item Developed software to detect spatially and temporally coherent features in 
                             ground motion data
                       \item Introduced novel techniques to infer the geophysical processes causing observable 
                             ground deformation          
                       \end{itemize}}\vspace*{-4mm}
\cvitem{2011}{\textbf{Ecological modeling intern}, \textit{Smithsonian Environmental Research Center}
              \begin{itemize}
              \item Used R and ArcGIS to create statistical models describing blue crab populations 
              	   in Chesapeake Bay   
              \end{itemize}}\vspace*{-6mm}       

\section{Teaching Experience}
\cvitem{2014--2016}{\textbf{Graduate student instructor}, \textit{University of Michigan}
                       \begin{itemize}
                       \item Instructed labs for a graduate level course on data analysis and 
                             inverse theory
                       \item Topics included statistical hypothesis testing, uncertainty quantification with 
                             bootstrapping, and model selection techniques such as cross validation and Bayesian 
                             information criterion. 
                       \end{itemize}}\vspace*{-4mm}
                               
\cvitem{2015}{\textbf{Michigan Math and Science Scholars instructor}, \textit{University of Michigan}
              \begin{itemize}
              \item Co-taught a summer course for high school students on the mathematics of natural 
                    hazards
              \end{itemize}}\vspace*{-6mm}
               
\section{Proficiencies}
\cvitem{}{\begin{itemize}
                  \item Geospatial analysis using free software packages such as Basemap and Shapely
                  \item Bayesian statistics and supervised machine learning
                  \item Python, Cython, MATLAB, R, Bash, \LaTeX
                  \end{itemize}}\vspace*{-4mm}



\section{Recent Publications}
\cvitem{2017}{Hines, T. T., and E. A. Hetland. Unbiased characterization of noise in geodetic data. submitted to \textit{Journal of Geodesy}}
\cvitem{2016}{Hines, T. T., and E. A. Hetland. Rheologic constraints on the upper mantle from 
              five years of postseismic deformation following the El Mayor-Cucapah earthquake. 
              \textit{J. Geophys. Res.}, 121, doi: 10.1002/2016JB013114}
\cvitem{2016}{Hines, T. T., and E. A. Hetland. Rapid and simultaneous estimation of fault slip 
              and heterogeneous lithospheric viscosity from post-seismic deformation. 
              \textit{Geophys. J. Int.}, 204, doi: 10.1093/gji/ggv477}  
            
\end{document}