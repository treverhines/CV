\documentclass[11pt,a4paper,sans]{moderncv} 

\moderncvstyle{casual} 
\moderncvcolor{blue} 

\usepackage[scale=0.78,top=1.0in,bottom=1.0in]{geometry}
%----------------------------------------------------------------------------------------
%   NAME AND CONTACT INFORMATION SECTION
%----------------------------------------------------------------------------------------

\firstname{Trever T.} % Your first name
\familyname{Hines} % Your last name
\address{2534 CC Little Bldg}{1100 North University Ave}{Ann Arbor, MI 48109-1005}
\phone{(217) 766 4445}
\social[github]{treverhines}
\email{hinest@umich.edu}

%----------------------------------------------------------------------------------------
\begin{document}

\makecvtitle % Print the CV title
\vspace*{-1cm}

\section{Education}
\cvitem{2012--present}{\textbf{PhD candidate Geophysics (certificate in Computational Engineering)},
                       \newline \textit{University of Michigan}, Ann Arbor, MI
                       \newline Graduation: August 16, 2017
                       \newline Advisor: Eric A. Hetland
                       \newline Dissertation: Transient ground deformation in tectonically active 
                       regions and implications for the mechanical behavior of the crust and upper mantle}
\cvitem{2008-2012}{\textbf{BS Geology}, \textit{University of Illinois}, Urbana-Champaign, IL}



\section{Research Experience}

\cvitem{2012--present}{\textbf{Graduate student research assistant}, \textit{University of Michigan}
			          \begin{itemize}
                       \item Developed a machine learning algorithm to detect geophysical signal in geodetic data 
                       \item Used the detected geophysical signal to improve our understanding of the Earth's  
                             composition and tectonic processes
                       \item Introduced an innovative method to characterize noise in geodetic data
                       \end{itemize}}\vspace*{-4mm}
\cvitem{2011}{\textbf{Ecological modeling intern}, \textit{Smithsonian Environmental Research Center}
              \begin{itemize}
              \item Used R and ArcGIS to create statistical models describing blue crab populations in Chesapeake Bay   
              \end{itemize}}\vspace*{-6mm}       

\section{Teaching Experience}
\cvitem{2014--2016}{\textbf{Graduate student instructor}, \textit{University of Michigan}
                       \begin{itemize}
                       \item Instructed labs for a graduate level course on data analysis and 
                             inverse theory
                       \item Topics included statistical hypothesis testing, uncertainty quantification, regression, and Bayesian inference
                       \end{itemize}}\vspace*{-4mm}
                               
\cvitem{2015}{\textbf{Michigan Math and Science Scholars instructor}, \textit{University of Michigan}
              \begin{itemize}
              \item Co-taught a course for high school students on the mathematics of natural hazards
              \end{itemize}}\vspace*{-6mm}
\section{Programming Proficiencies}
\cvitem{}{Python, Cython, MATLAB, R, Bash, \LaTeX}               
\section{Recent Publications}
\cvitem{2017}{Hines, T. T., and E. A. Hetland. Revealing transient strain in geodetic data with Gaussian process regression. \textit{submitted to Geophys. J. Int.}}
\cvitem{2017}{Hines, T. T., and E. A. Hetland. Unbiased characterization of noise in geodetic data. \textit{submitted to J. Geod.}}
\cvitem{2016}{Hines, T. T. and E. A. Hetland. Rheologic constraints on the upper mantle from five years of postseismic deformation following the El Mayor-Cucapah earthquake. \textit{J. Geophys. Res.}, 121, doi: 10.1002/2016JB013114}
\cvitem{2016}{Hines, T. T. and E. A. Hetland. Rapid and simultaneous estimation of fault slip and heterogeneous lithospheric viscosity from post-seismic deformation. \textit{Geophys. J. Int.}, 204, doi: 10.1093/gji/ggv477}
        
\end{document}